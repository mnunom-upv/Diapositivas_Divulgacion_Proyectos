\documentclass[12pt]{book}
\begin{document}
En la ecuacion \ref{ecuacion1}, 
se define el calculo de la variable Z
\begin{equation}
\label{ecuacion1}
Z = X + Y
\end{equation}
Podemos calcular de forma alterna lo siguiente: 
$\alpha = \beta + \gamma$
Subindices = $a_n = a_{n-1} + a_{n-2} + ... $
Fracciones: $y = \frac{x}{z-1} $
\begin{equation}
\label{ecuacion2}
y = \frac{x}{z-1}
\end{equation}
\begin{equation}
\label{ecuacion3}
promedio = \frac{\sum_{i=0}^{n}x_i}{n}
\end{equation}
\begin{equation}
\label{ecuacion4}
integral = \int_{i=0}^{n}x_i
\end{equation}
\end{document}
