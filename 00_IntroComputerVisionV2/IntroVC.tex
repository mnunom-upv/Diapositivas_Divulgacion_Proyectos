
\begin{frame}{Visión por Computadora}
%\begin{block}{Fundamentación} 
\begin{columns}
\begin{column}{0.60\textwidth}
    \begin{center}
\begin{itemize}
\item La visión por computadora imita la percepción humana y las capacidades de razonamiento.
\item Se intenta inferir conocimiento a partir de imágenes o videos.
\item Procesamiento de imágenes es una fase temprana de la VC. La entrada es una imagen y la salida es otra imagen.
\item En la visión por computadora, la entrada es una imagen pero la salida son datos. 
\end{itemize}
     \end{center}

\end{column}
\begin{column}{0.40\textwidth}  
    \begin{center}
     \includegraphics[width=0.9\textwidth]{Figs/VC}
     \end{center}
\end{column}
\end{columns}
%\end{block} 
\end{frame}

\begin{frame}{Visión por Computadora (Template Matching)}

Permite comparar imágenes aplicando operaciones de comparación entre regiones.
\begin{columns}
  \column {0.5\textwidth}
  \begin{itemize}
    \item Algoritmos clásicos basados en correlación: SAD, SSD, Rank.
	  \begin{itemize}
		    \item Ventajas: Facil de entender, rápidos
		    \item Desventajas: Sensibles a la escala, a rotaciones y a cambios en la iluminación. 
	  \end{itemize}
    \item Algoritmos modernos basados Aprendizaje Profundo
		\begin{itemize}
		    \item Ventajas: Buen desempeño 
		    \item Desventajas: Requieren de entrenamiento con muchas imágenes
	  \end{itemize}

  \end{itemize}
    \column {0.5\textwidth}  
        \begin{center}
            \includegraphics[width=\textwidth]{00_IntroComputerVision/figs/Industrial-software-example-for-Template-Matching_W640}\\
     \end{center}

    \end{columns}
\end{frame}


