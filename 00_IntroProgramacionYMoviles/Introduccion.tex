

\begin{frame}
\frametitle{Computadora y Programabilidad}  
\begin{block}{Computadora}
Es una m\'aquina (electr\'onica) programable* que recibe y procesa datos para convertirlos en informaci\'on \'util. Contiene perif\'ericos de entrada (para introducir datos) y salida (para mostrar resultados) \pause
\end{block}
\begin{block}{Algoritmo}
Conjunto finito de instrucciones para resolver una tarea espec\'ifica \pause
\end{block}

\begin{block}{Programaci\'on}
El proceso de crear un software utilizando un lenguaje de programacion (C, C++, Java, Python, Kotlin, etc) \pause
\end{block}

\begin{block}{Programa}
Un conjunto de instrucciones que una computadora interpreta en una secuencia logica para llevar a cabo una tarea en particular \pause
\end{block}

\end{frame}

\begin{frame}
\frametitle{Diagrama de Flujo} 
\begin{columns}
\column{0.5\linewidth}
Un diagrama de flujo es una representaci\'on gr\'afica de un algoritmo, los pasos que los componen y la secuencia de ejecuci\'on de sus instrucciones
\begin{block}{Algoritmo}
Dise\~nar un algoritmo para comparar dos numeros
\end{block}

\column{0.5\linewidth}
\begin{center}
\includegraphics[width=0.95\linewidth]{00_IntroProgramacionYMoviles/DiagramaFlujo1.jpg} 
\end{center}
\end{columns}
\end{frame}

\begin{frame}
\frametitle{Codificacion de un algoritmo en varios lenguajes de programaci\'on (Python - Kotlin)} 
\begin{columns}
\column{0.45\linewidth}
\begin{block}{Codificacion del Algoritmo en Python}
\inputminted[linenos,fontsize=\tiny]{python}{00_IntroProgramacionYMoviles/Hello.py}
\end{block}
\column{0.45\linewidth}
\begin{block}{Codificacion del Algoritmo en Kotlin}
\inputminted[linenos,fontsize=\tiny]{python}{00_IntroProgramacionYMoviles/Hello.kt}
\end{block}
\end{columns}
\end{frame}



\begin{frame}
\frametitle{Sistema Operativo}  
\begin{columns}
%\column{0.32\linewidth}
\column{0.65\linewidth}
\begin{block}{}
Un Sistema Operativo (SO) es un programa (software) que al arrancar la computadora** se encarga de gestionar todos los recursos del sistema informático permitiendo así la comunicación entre el usuario y la computadora. 
\end{block}
\begin{center}
\includegraphics[width=0.95\linewidth]{00_IntroProgramacionYMoviles/SistemaOperativo1.png} 
\end{center}
\tiny{\url{https://reader.digitalbooks.pro/content/preview/books/38230/book/OEBPS/Text/c1.html}}

\column{0.32\linewidth}
\begin{center}
\includegraphics[width=0.95\linewidth]{00_IntroProgramacionYMoviles/SistemaOperativo2.png} 
\end{center}
\end{columns}

\end{frame}


\begin{frame}
\frametitle{Sistemas Operativos para PCs}  

\begin{columns}
\column{0.32\linewidth}
\begin{center}
\includegraphics[width=0.95\linewidth]{00_IntroProgramacionYMoviles/Windows11.png} 
\end{center}

\column{0.32\linewidth}
\begin{center}
\includegraphics[width=0.95\linewidth]{00_IntroProgramacionYMoviles/MacOS.png} 
\end{center}

\column{0.32\linewidth}
\begin{center}
\includegraphics[width=0.95\linewidth]{00_IntroProgramacionYMoviles/Linux.png} 
\end{center}
\end{columns}

\end{frame}





\begin{frame}
\frametitle{Telefono Celular No-inteligente vs Telefono Celular Inteligente}  

\begin{columns}
\column{0.46\linewidth}
\begin{block}{Tel\'efono No-inteligente}
\begin{itemize}
\item Su funcionalidad principal era la comunicaci\'on (llamadas o mensajes) a trav\'es de la red celular (GSM)
\end{itemize}
\end{block}
\begin{block}{Tel\'efono inteligente}
\begin{itemize}
\item Interfaz de entrada: Pantalla Touch (a color, de alta definici\'on) 
\item Conexi\'on a Internet: WiFi, GSM (4G o 5G)
\item Comunicaci\'on con otros dispositivos: Bluetooth, NFC
\item C\'amaras (Frontal y Posterior)
\end{itemize}
\end{block}

\column{0.18\linewidth}
\begin{center}
\includegraphics[width=0.95\linewidth]{00_IntroProgramacionYMoviles/FeaturePhone_Nokia.png} 
\end{center}
\column{0.28\linewidth}
\begin{center}
\includegraphics[width=0.95\linewidth]{00_IntroProgramacionYMoviles/Smartphone_Motorola.png} 
\end{center}
\end{columns}
\end{frame}

\begin{frame}
\frametitle{Sistemas Operativos para Telefonos Inteligentes} 
\begin{columns}
\column{0.32\linewidth}
\begin{center}
\includegraphics[width=0.95\linewidth]{00_IntroProgramacionYMoviles/Android.png} 
\end{center}

\column{0.32\linewidth}
\begin{center}
\includegraphics[width=0.95\linewidth]{00_IntroProgramacionYMoviles/iOs.png} 
\end{center}

\column{0.32\linewidth}
\begin{center}
\includegraphics[width=0.95\linewidth]{00_IntroProgramacionYMoviles/WindowsPhone.png} 
\end{center}
\end{columns}
 
\end{frame}


\begin{frame}
\frametitle{Android} 
\begin{columns}
\column{0.64\linewidth}
\begin{itemize}
\item Android es un sistema operativo móvil basado en Linux 
\item Principalmente orientado a dispositivos de pantalla t\'actil (Smartphone, tablets, smartwatches, etc)
\item Fue desarrollado por Android Inc (Adquirida por Google en 2005)
\item Vinculado con un grupo de empresas (HTC, Sony, Motorola, Samsung, LG, Lenovo, entre otras) para la creaci\'on de un SO com\'un para sus dispositivos
\item A la fecha (Q1 2023), los tel\'efonos con SO Android concentran mas del 70\% del mercado global. 
\end{itemize}
\column{0.32\linewidth}
\begin{center}
\includegraphics[width=0.95\linewidth]{00_IntroProgramacionYMoviles/AndroidVSIOs_WorldWide.png} 
\end{center}
\end{columns}
 
\end{frame}





\begin{frame}
\frametitle{Aplicaciones Móviles}  
\begin{columns}
\column{0.4\linewidth}
\begin{itemize}
\item Ejecutadas en el tel\'efono
\item La entrada de datos es mediante un teclado ``virtual''
\item El apuntador del raton es la pantalla 
\item Incluyen una interfaz de usuario gr\'afica (GUI) 
\item Es posible descargar miles de \'estas en nuestros dispositivos
\end{itemize}
\column{0.30\linewidth}
\begin{center}
\includegraphics[width=0.95\linewidth]{00_IntroProgramacionYMoviles/TiposAplicaciones.png} 
\end{center}
\column{0.30\linewidth}
\begin{center}
\includegraphics[width=0.95\linewidth]{00_IntroProgramacionYMoviles/most-popular-apps.jpg} 
\tiny{\url{https://www.netsolutions.com/insights/top-10-most-popular-apps-2018/}}  
\end{center}
\end{columns}
\end{frame}


\begin{frame}
\frametitle{Android Studio}  

\begin{itemize}
\item Android Studio es un entorno oficial de desarrollo integrado (IDE) para el sistema operativo Android de Google
\item La primera versi\'on se libera en el año 2013, siendo el lenguaje de programacion Java
\item En 2019, se reemplaza el lenguaje oficial de desarrollo por Kotlin, aunque Java todav\'ia es soportado
\item Es gratis, se puede descargar e instalar en cualquier computadora sin importar el sistema operativo (Windows, Linux y MacOS)
\url{https://developer.android.com}
\end{itemize}
\end{frame}
